\documentclass[webpdf,contemporary,large,namedate]{oup-authoring-template}%

%\PassOptionsToPackage{hyphens}{url}
%\PassOptionsToPackage{colorlinks,linkcolor=blue,urlcolor=blue,citecolor=blue,anchorcolor=blue}{hyperref}

\DeclareMathOperator*{\argmax}{argmax}

\usepackage{algorithmicx}
\usepackage{lmodern}
\usepackage{setspace}
\renewcommand{\ttdefault}{cmtt}

\begin{document}
\journaltitle{TBD}
\DOI{TBD}
\copyrightyear{2025}
\pubyear{2025}
\access{Advance Access Publication Date: Day Month Year}
\appnotes{Preprint}
\firstpage{1}

\title[Finding low-complexity regions]{Identifying low-complexity regions with longdust}
\author[1,2,3,$\ast$]{Heng Li\ORCID{0000-0003-4874-2874}}
\author[4]{Brian Li}
\address[1]{Department of Biomedical Informatics, Harvard Medical School, 10 Shattuck St, Boston, MA 02215, USA}
\address[2]{Department of Data Science, Dana-Farber Cancer Institute, 450 Brookline Ave, Boston, MA 02215, USA}
\address[3]{Broad Insitute of MIT and Harvard, 415 Main St, Cambridge, MA 02142, USA}
\address[4]{Commonwealth School, Boston, MA 02116, USA}
\corresp[$\ast$]{Corresponding author. \href{mailto:hli@ds.dfci.harvard.edu}{hli@ds.dfci.harvard.edu}}

%\received{Date}{0}{Year}
%\revised{Date}{0}{Year}
%\accepted{Date}{0}{Year}

\abstract{
\sffamily\footnotesize
\textbf{Background:}
\vspace{0.5em}\\
\textbf{Results:}
\vspace{0.5em}\\
\textbf{Conclusion:}
\vspace{0.5em}\\
\textbf{Availability and implementation:}
\url{https://github.com/lh3/longdust}
}

\maketitle

\section{Introduction}

\section{Methods}

\subsection{Notations}

Let $\Sigma=\{{\tt A},{\tt C},{\tt G},{\tt T}\}$ be the DNA alphabet,
$x\in\Sigma^*$ is a DNA string and $|x|$ is its length.
$t\in\Sigma^k$ is a $k$-mer.
For $|x|\ge k$, $c_x(t)$ is the occurrence of $k$-mer $t$ in $x$;

$\ell(x)=\sum_t c_x(t)=|x|-k+1$ is the total number of $k$-mers in $x$.
Introduce $\vec{c}_x$ as the count array over all $k$-mers
and $\kappa(x)$ as the set of distinct $k$-mers in $x$.

In this article, we assume there is one long genome string.
We use closed interval $[i,j]$ to represent the substring starting at $i$
and ending at $j$, including the end points $i$ and $j$.
We may use ``interval'' and ``subsequence'' interchangeably.

\subsection{Modeling $k$-mer counts}

Suppose symbols in $\Sigma$ all occur at equal frequency.
Then $c_x(t)\sim{\rm Poisson}(\lambda)$ where $\lambda=\ell(x)/4^k$.
Let
$$
p(n|\lambda)\triangleq\frac{\lambda^n}{n!}e^{-\lambda}
$$
be the probability mass function of Poisson distribution.

The composite probability of string $x$ can be modeled by
$$
P(x)=\prod_{t\in\Sigma^k}p(c_x(t)|\lambda)
$$
We have
$$
\log P(x)=4^k\lambda(\log\lambda-1)-\sum_t\log c_x(t)!
$$
The average of $\log P(x)$ can be approximated as
\begin{eqnarray*}
H(\lambda)&\triangleq&\sum_{\vec{c}}P(\vec{c})\cdot\sum_t\log p(c_t|\lambda)\\
&=&4^k\sum_{n=0}^{\infty}p(n|\lambda)\log p(n|\lambda)\\
&=&4^k\lambda(\log\lambda-1)-4^ke^{-\lambda}\sum_{n=0}^{\infty}\log n!\cdot\frac{\lambda^n}{n!}
\end{eqnarray*}
which is the negative entropy of $P$. Note that in theory $n\le\ell(x)$,
but with $\ell(x)\gg1$ in practice,
$$
p(\ell|\lambda)\approx\frac{e^{-\lambda}}{\sqrt{2\pi\ell}}\cdot\left(\frac{e}{4^k}\right)^{\ell}\ll 1
$$
under Sterling's approximation.
The sum to $\infty$ in $H(\lambda)$ is justified.
Define
$$
Q(x)\triangleq H(\lambda)-\log P(x)=\sum_t\log c_x(t)!-f\left(\frac{\ell(x)}{4^k}\right)
$$
where
$$
f(\lambda)\triangleq4^ke^{-\lambda}\sum_{n=0}^\infty\log n!\cdot\frac{\lambda^n}{n!}
$$

\subsection{The $k$-mer repetitiveness of a string}

We measure the repetitiveness of string $x$ by $S(\vec{c}_x)$,
which is a function of $k$-mer counts but does not depend on other properties of $x$.
For convenience, we may also write $S(x)$ as the score function.
We say $x$ is a \emph{perfect string/interval} if no substring of $x$ is scored higher than $S(x)$;
$x$ is a \emph{good string/interval} if no prefix or suffix of $x$ is scored higher than $S(x)$.



\section{Results}

\section{Discussions}

\section*{Acknowledgments}

\section*{Author contributions}

H.L. conceived the project, implemented the method, analyzed the data and drafted the manuscript.
B.L. prototyped the algorithm.

\section*{Conflict of interest}

None declared.

\section*{Funding}

This work is supported by National Institute of Health grant R01HG010040, U01HG013748 and U41HG010972 (to H.L.).

\section*{Data availability}

\url{https://github.com/lh3/longdust}

\bibliographystyle{apalike}
{\sffamily\small
\bibliography{longdust}}

\end{document}
